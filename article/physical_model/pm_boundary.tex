\subsection{Boundary conditions}
For Poisson's equation, at the charged boundaries, most physical
situations may be covered by either specifying the potential or the
surface charge density. The former would be a Dirichlet boundary
condition

\begin{equation}
\psirm(\x) = \zeta(\x)\;,\;\; \x \in \Gamma
\end{equation}
and the latter a Neumann condition 

\begin{equation}\label{eq:et:fix_c}
\nabla\psirm(\x) \cdot \n =
-\frac{\sigma(\x)}{\epsilon_0\epsilon_r}\;,\;\; \x \in \Gamma
\end{equation}
where $\Gamma$ denotes the boundary of the domain and $\n$ is the
normal to the boundary surface.\- \cite{hlushkou} In this paper,
simulations are performed with a fixed surface charge.

At hard boundaries (walls), the charge flux through the boundary is
set to zero, i.e.

\begin{equation}\label{eq:et:j0}
\J \cdot \mathbf{n} = 0 \;,\;\; \x \in \Gamma
\end{equation}
where $\mathbf{n}$ denotes the normal to the surface and $\Gamma$ is
the boundary of the domain.

The boundary condition of the velocity field is chosen as a friction
condition, i.e.

\begin{equation}
\ubf = 0 \;,\;\; \x \in \Gamma
\end{equation} 

At wet boundaries, inlets and outlets, of the domain various boundary
conditions may be set. For instance the pressure or the velocity could
be fixed. In the case of a fixed pressure boundary, a flow direction
must also be specified for completeness. \cite{he_zou}
