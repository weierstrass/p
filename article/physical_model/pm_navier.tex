\subsection{The velocity field - Navier-Stokes equations}\label{sec:et:ns}
The Navier-Stokes equations are among the most fundamental corner
stones of hydrodynamics. They describe the motion of a fluid under
the influence of various internal and external forces.

For later convenience and for reference when it comes to deriving the
Lattice-Boltzmann formulation of the NS equation, a brief sketch of a
derivation will here be presented. A most general form of the
Navier-Stokes equation follows from momentum conservation

\begin{equation}
\dfrac{\partial (\rhorm \ubf)}{\partial t} + \nabla \cdot (\rho \ubf
\otimes \ubf) + \Q = 0 
\end{equation}
where, $\rhorm$ is fluid density, $\ubf$ is velocity, $\otimes$
represents the outer product and $\Q$ is a momentum source term
(force per volume). Expanding the time derivative and the divergence
terms respectively gives
 
\begin{equation}\label{eq:et:nspre}
\ubf \left ( \dfrac{\partial \rhorm}{\partial t} + \nabla \cdot
  (\rhorm \ubf) \right ) + \rhorm \left (\dfrac{\partial \ubf}{\partial t} +
  \ubf \cdot \nabla \ubf 
  \right ) + \Q = 0.
\end{equation}
To assure mass conservation (without sources) we have

\begin{equation}\label{eq:et:mass_conc}
 \dfrac{\partial \rhorm}{\partial t} + \nabla \cdot(
  \rhorm \ubf) = 0
\end{equation}
and eq. \eqref{eq:et:nspre} reduces to

\begin{equation}\label{eq:et:ns_general} 
\rhorm \left (\dfrac{\partial \ubf}{\partial t} +
  \ubf \cdot \nabla \ubf 
  \right ) + \Q = 0
\end{equation}
which together with eq. \eqref{eq:et:mass_conc} is a general
formulation of the Navier stokes equations. 

The force term $\Q$, is determined by the physical properties of the
fluid and from its environment. In this work, only incompressible
($\rho = \mbox{constant}$) Newtonian fluids will be studied. The force
contribution to $\Q$ involved in that case is limited to viscous
forces, pressure gradients in the fluid and to external force
fields. Putting this into eqs. \eqref{eq:et:mass_conc} and
\eqref{eq:et:ns_general} gives

\begin{equation}\label{eq:et:ns_incompressible}
 \nabla \cdot \ubf = 0
\end{equation}
and

\begin{equation}\label{eq:et:ns_mom}
\rhorm \left (\dfrac{\partial \ubf}{\partial t} +
  \ubf \cdot \nabla \ubf 
  \right ) = - \nabla \Prm  + \mu \nabla^2 \ubf + \F
\end{equation}
where $\Prm$ is the pressure, $\mu$ the kinematic viscosity and $\F$
is the contributions from  external forces.

\subsubsection{Boundary conditions}

At hard boundaries (walls), the boundary conditions to
eqs. \eqref{eq:et:ns_incompressible} and \eqref{eq:et:ns_mom} are set
on the velocity of either a Dirichlet or Neumann type. In most
physical situations the Dirichlet condition is used which corresponds
to that there is a friction between the fluid and the wall, usually
full friction, i.e. when no relative movement between fluid and wall
is present and the velocity at the wall boundary is set to zero, i.e.

\begin{equation}
\ubf = 0 \;,\;\; \x \in \Gamma
\end{equation} 
where $\Gamma$ denotes the boundary. The Neumann type conditions are
used for no-friction walls where the normal component of the
derivative of the velocity is specified, usually to zero.

At wet boundaries, inlets and outlets, of the domain various boundary
conditions may be set. For instance the pressure or the velocity could
be fixed. In the case of a fixed pressure boundary, a flow direction
must also be specified for completeness. \cite{he_zou}
