\section{Physical model}\label{sec:pm}

Electrohydrodynamics involves the study of electric phenomena on fluid
flow. How fluids carrying electrical charges (electrolytes) react upon
external electrical fields or interact with charged objects are
examples of problems that arise in this field. 

As a charged object is brought into contact with an electrolyte it is,
qualitatively, easily deduced that ions with a sign of charge opposite
to that of the object will be attracted to the object and ions with
the same sign of charge will be repelled. These two distinct
categories of ions will from hereon be referred to as counter- and
co-ions respectively. In this case, for a neutral electrolyte, a
surplus of counter-ions will be present in the direct vicinity of the
object and a surplus of co-ions will be present at some other location
further from the object. The part of the net charged fluid nearest to
the charged boundary is often referred to as an electrical double
layer (EDL) \cite{dongquing-ren-book}. 

Traditionally, the physical description of EDLs and electrokinetics in
general is based on the Poisson-Boltzmann model. However the PB model
includes some crude assumptions on the system and is sometimes not an
accurate choice, see section \ref{sec:et:pb} for a brief
description. Therefore, more sophisticated modelling approaches has
been proposed e.g. in \cite{}. This paper is based on such an approach
which is described further in this section.

\input{physical_model/pm_coupling.tex}

\subsection{The potential - Poisson's equation}\label{sec:et:poisson}
To be able to model the flow dynamics of liquids in a channel with
present EDLs, the potential and charge distribution
in the channel must be determined. These quantities are mutually
related through Poisson's equation for electrostatics:

\begin{equation}\label{eq:pb}
\nabla^2\psirm = -\frac{\rho_e}{\epsilon_r \epsilon_0}
\end{equation}
where $\psirm$ is the electrical potential, $\rho_e$ the electrical
charge density, $\epsilon_r$ is the relative permittivity and
$\epsilon_0$ the vacuum permittivity. Under certain assumptions, the
charge density may be explicitly determined as a function of the
potential distribution, one such result is the so called
Poisson-Boltzmann equation, further discussed in section \ref{sec:et:pb}.


\input{physical_model/pm_np.tex}
\input{physical_model/pm_pb.tex}

\input{physical_model/pm_navier.tex}

\input{physical_model/pm_electrovis.tex}

\input{physical_model/pm_electroosmosis.tex}

%lite intro

%börja inte med allmän teorpmisk formulering

%sedan lite resultat, double layers

%pressure driven flows...
