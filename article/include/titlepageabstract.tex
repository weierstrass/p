\begin{center}
\LARGE \textbf{Modelling of electrokinetic flow\\ using the
  lattice-Boltzmann method} \normalsize

\vspace{10pt}
\line(1,0){100}\\
\vspace{16pt}
\large Andreas Bülling and Alexei Heintz
\end{center}

\vspace{10pt}

\begin{abstract}
\todo{Update abstract!}
The lattice-Boltzmann method is used to model flow in electrokinetic
systems. A modelling approach based on the coupling of Navier-Stokes,
Nernst-Planck and Poisson's equation of electrostatics is
utilised. Three lattice-Boltzmann methods are formulated for the three
equations respectively.

The method is implemented in \texttt{C++} with the aim of being high
performing. Topics as locality, instruction pipelines and parallel
computing are considered. The implementation is tested for a number of
classic examples with known solutions, e.g.  Taylor-Green vortex flow,
an Helmholtz equation and an advection-diffusion situation. The
computed solutions agree well with the analytic solutions.

The physical systems modelled consists mainly of various charged
channel flows of ionic solutions. Electrokinetic effects, such as
electroosmosis and the electrovicous effect are studied. This is done
in thin channels where the thickness of the electrical double layers
is comparable to the channel dimension. The electroviscous effect is
shown to slow the flow down and a local minimum is found in the
velocity profile for thick enough double layers. Other more
complicated systems are also studied; electroosmotic flow in a channel
with heterogeneously charged walls and flow in a an array of charged
squares.
\end{abstract}

\noindent \textbf{Keywords:} lattice-Boltzmann, electrokinetics,
electrohydrodynamics, Nernst-Planck, Poisson-Boltzmann,
high performance computing.

