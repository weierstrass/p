\section{Chapman-Enskog vs. regular expansion analysis}
When reading literature and articles about the LBM, different
approaches may be used to derive the macroscopic behaviour of the
LBM. The aim of this section is to bring some clarity and briefly
explain the main differences between two of these.

The method used in this work is referred to as regular (error)
expansion analysis \cite{junk-comparison} and is used with one time
scale. In the case of analysing the macroscopic behaviour of the LBM
for the Navier-Stokes equations, the incompressible equations are
obtained. The mass and momentum equations are exactly satisfied by
$\rhoe{0}$ and $\ue{1}$ from the regular expansions of $\rho$ and
$\ubf$ respectively.

An other approach is by using the so called Chapman-Enskog analysis
\cite{wolf-gladrow}. This is a traditional method in kinetic theory
and is the most frequently used in litterature about the LBM. Here two
time scales are used, on faster (convective) and one slower
(diffusive). This gives that in the macroscopic limit of the LBE, the
compressible Navier-Stokes equations are obtained. In the C-E analysis,
the full quantities $\rho$ and $\ubf$ are showed to satisfy, not the
exact, but the N-S equations with higher order error terms.

